After analysing the three signals with the DFT algorithm we come to the conclusion that discrete signals simply cannot be processed without being affected by spectral leakage. It does not matter if the number of frequency bins is equal to the sampling frequency, there will always be non-zero magnitudes at frequency bins that should be at 0 (at the least in the practical sense). Despite this, the peaks were still recognizable and the spectrums displayed the correct frequencies at the right bin frequencies.\\

This leads to the conclusion that DFT is not supposed to be used in any general application. It needs to be configured specifically for a certain application, with all trade offs being considered. This also includes windowing, which has demonstrated to improve the output quite considerably.\\

That is the great benefit of DSP and mathematics in general: if a signal is not being processed adequately, just change the number of DFT points or select a different type of window, since there are so many. If no window fits the given application, then it's still possible to design a custom window, because after all, the DFT and Windowing functions are nothing more than just Z transforms and a set of coefficients.