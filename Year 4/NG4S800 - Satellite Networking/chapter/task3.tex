\large\textbf{a)} \normalsize
Based on the ITU-T specification, an STM-1 signal may be constructed from an E3 signal by first adding a POH (Path Overhead) field next to the STM-1 payload in the 10th frame column, which places the C3 signal into a virtual container in order to identify and manage the transmission from source to destination using the byte Path Trace and BIP-8 code to check for TX/RX errors. In simpler terms, the POH converts the C3 signal into a VC-3. This step is called \textbf{mapping}.\\

Afterwards, a pointer is inserted in the STM-1 header between the RSOH (Regenerator Section Overhead) and MSOH (Multiplex Section Overhead) fields on the 4th header row so that the C3 data can be accessed within the payload field using the correct offset. The output of this stage produces the AU-3 (Administrative Unit) signal. Also, this step is called \textbf{pointer processing} or \textbf{alignment}.\\

Next, three AU-3 signals from three different sources are multiplexed with byte-interleaving into one single output signal named AUG (Administrative Unit Group).\\

Finally, the STM-1 frame is completed after adding the SOH (Section Overhead) fields, composed of the RSOH and MSOH fields. These are meant to be used in-between regenerators and multiplexing points. The RSOH contains information about frame alignment, as well as a channel for inter-regenerator communication (through byte E1) and error checking, such as RS trace message and BIP-8 code. Similarly, MSOH contains bits for error checking with BIP-24 and another E2 field for communication between two multiplexing points.\\

Following these steps should result in the diagram:
\showimage{itu-t-c3}{Building an STM-1 frame from three E3 signals}

The efficiency is:\\
$\rightarrow\text{If the STM-1 signal contains 3 E3 signals;}$\\
$\rightarrow\text{And each E3 signal contains } 4\times E2 = 4^2\times E1 = 4^2\times 30\times E0 = 480 \text{ voice channels;}$\\
$\rightarrow\text{Each of which has a bit rate of 64 kb/s, thus: }$
$$Bitrate_{Payload} = 3\times E3 = 3\times 480\times 64 \text{ kb/s} = 92160 \text{ kb/s}$$
$\rightarrow\text{Therefore, the efficiency must be equal to:}$
$$\text{Efficiency}_{\text{STM-1}} = \frac{Bitrate_{Payload}}{Bitrate_{\text{STM-1}}}\times 100\% = \frac{92160\text{ kb/s}}{155520 \text{ kb/s}}\times 100\% = \mathbf{59.26}\%$$

$\rightarrow\text{As opposed to the E3 efficiency:}$\\
$$\text{Efficiency}_{\text{E3}} = \frac{480 \text{ voice channels}\times 64\text{ kb/s}}{34368\text{ kb/s}}\times 100\% = \mathbf{89.39}\%$$

As can be seen, the STM-1 efficiency is quite low when compared to the PDH E3. This is because a single E3 is not capable of filling up the entire payload space on the STM-1 frame, therefore, the STM-1 transmission channel will carry a lot of overhead data just for three multiplexed E3 signals.\\

\large\textbf{b)} \normalsize
SDH allows the transmission of a PDH signal within its payload by allocating extra space in this same field and by allowing the PDH data to vary its offset position with respect to the payload's starting point on column 10 of the STM-1 frame structure. (Note that when it is said \quotes{allocating extra space}, it is meant \quotes{a space that is larger than a single PDH frame}. The STM-1 payload field is fixed at $261 \text{ columns}\times 9 \text{ rows} = 2349$ bytes of size and does not grow in respect to the PDH signal being conveyed. If one STM-1 frame cannot fit enough PDH signals, then an STM-N of N $>$ 1 should be used instead.)\\

This way, the bit rate variance property that characterises the nature of the PDH signal is conserved and the SDH signal, being a synchronous transmission, will still be able to carry a plesiosynchronous signal. This is also why there is a pointer offset in the STM-1 header that varies its value depending on the payload's bit rate. It is used to provide access to the virtual container at the right offset within the payload field.\\

\large\textbf{c)} \normalsize
TCP was specially designed to handle transmission of IP packets between two hosts ideally through the coaxial or fiber medium. This protocol handles congestion in these physical mediums quite well as opposed to the wireless medium, where it performs considerably worse. The reason is the fact that in these first two mediums the relative distances between two hosts is (most of the time) quite short. In satellite networks however, the distances are considerably larger, with LEO orbit altitudes reaching up to 5000 km, MEO 5000 - 20000 km and HEO above 20000 km\autocite{book1_pp30}. The \textbf{absolute minimum} time it takes for radio waves to travel to the LEO orbit is $\frac{5000 \text{ km}}{c} = \frac{5\cdot 10^6}{3\cdot 10^8} = 1.7\cdot 10^{-2} \text{ s} = 17 \text{ ms}$, and in the HEO orbit the travel time is $\frac{20000 \text{ km}}{c} = \frac{20\cdot 10^6}{3\cdot 10^8} = 6.7\cdot 10^{-2} \text{ s} = 67 \text{ ms}$. This is only considering the speed of light with the best case scenario where there are no extra delays. In other words, TCP works well on copper and fiber because the RTT (Round Trip Time) is in the order of milliseconds and timeout issues are much less of a problem, whereas in the wireless medium, such as the communication between a base station and a geostationary satellite, the RTT is considerably larger to the point where it takes more than one second for a packet to take a single hop (with all the delays being accounted for).\\

The main issue is not necessarily that the packet or segment takes a very long time to reach its destination. (Since this issue is unavoidable, unless the satellite is moved to a lower orbit from HEO/GEO to MEO/LEO.) The true problem lies in the fact that TCP expects the segment to reach its destination within a negotiated amount of time. If the transmission takes too long, a timeout event will occur and the segment \squotes{growth rate} (SSThresh) will begin to lower. If the timeout keeps occurring, the client might stop receiving any packet at all and it will stay in the initial mode Slow Start indefinitely until the RTT is lowered or the TCP header is optimized for the specific link. Another issue found in satellite networks is the inability to utilise the whole bandwidth provided by the link. Essentially, TCP defines a parameter in its segment header that limits the maximum data size that is received by the client, called the window parameter. This value depends on many things, mainly on what the client is capable or willing to receive.\\

Taking this into account, in order to calculate the link utilisation, a critical parameter must first be calculated: the BDP - Bandwidth-delay Product.
$$BDP = RTT\times Bandwidth$$
This represents the number of data bits that are \squotes{currently} travelling in a medium which have not yet been acknowledged. So, if the window size only allows $2^{16 \text{ bits}} - 1 = 64$ kB to be sent (this size is defined by the TCP protocol in the header structure) and the client is requesting a total amount of data of 128 kB, then the entire \squotes{transaction} should theoretically take $\tfrac{128 \text{ kB}}{64 \text{ kB}} = 2$ RTTs to complete. This will negatively impact the congestion-control algorithm, especially if the RTT is larger than 500 ms. Moreover, the link utilisation will be (assuming $WindowSize = 2^{16}-1 = 65535 = 64$ kB, $RTT = 500$ ms and $Bandwidth = 3$ Mbit/s):
$$\text{Link Utilisation (\%)} = \dfrac{WindowSize}{BDP}\times 100\% = \dfrac{65535\times 8}{0.5\times 3\cdot10^6}\times 100\% = \mathbf{35}\%$$

Another way of getting the utilisation is by calculating the maximum ideal throughput:
$$Throughput = \dfrac{WindowSize}{RTT} = \dfrac{65535\times 8}{0.5} = \mathbf{1048560} \text{ b/s}$$

Which, when divided with the total bandwidth gives:
$$\text{Link utilisation (\%)} = \dfrac{Throughput}{Bandwidth}\times 100\% = \dfrac{1048560}{3\cdot 10^6}\times 100\% = \mathbf{35}\%$$

So, the first solution to improve the congestion-control algorithm, link utilisation and throughput is to simply increase the window size. This can be done in the TCP header in the \squotes{TCP Options} field by setting the first byte to a fixed value 3 (which indicates the operation \squotes{set window scale value}), the second byte to value 3 as well (the length of the entire option in bytes) and the third byte to an arbitrary value that ranges from 0 to 14. This third parameter controls the size of the window, scaling it with the bitwise left shift operation. Hence, the new window size is: $$WindowSize = TCPHeader_{WindowField} << TCPHeader_{OptionsField[2]}$$

Where $TCPHeader_{OptionsField[2]}$ is the 3rd byte with the scaling value that is in the range $[0, 14]$. This means that if $TCPHeader_{WindowField}$ is at its maximum value ($2^{16\text{ bits} - 1} = 65535$) and the option field is set to 14, the maximum obtainable window size that can be used in any network is $65535 << 14 = 1073725440$ bytes. Coming back to the previous example, a window of this size would allow the client to utilise 100\% of its available bandwidth since link utilisation would be $\frac{1073725440\times 8}{0.5\times 3\cdot10^6} > 100\%$ and the entire transmission would be concluded within 1 RTT (ideally, assuming that the transmission algorithm is already in the congestion avoidance mode and the segment size has reached the window size).\\

There are other solutions implemented for satellite networks to further improve the algorithm and the utilisation percentage\autocite{book1_pp269}. Some of the techniques are:

\begin{itemize}
\item 1- Increase the initial SSThres value which controls the point at which the TCP protocol switches from Slow Start mode to Congestion Avoidance mode, for the same reasons presented above;
\item 2- Also increase the minimum size ($T_b$) of the granular traffic segment block that comprises an entire segment;
\item 3- Use fast retransmission and fast recovery in order to allow the host to reach the Congestion Avoidance mode at a higher rate after a loss/timeout;
\item 4- Compress the TCP header structure to reduce the entire segment size\autocite{book2_pp355};
\item 5- Detect when a corruption loss occurs in a transmission;
\item 6- Connect to multiple satellites instead of only one in order to \squotes{multiplex} the RX data stream\autocite{book2_pp355}.
\end{itemize}