\paragraph{}
Satellite networks are composed of three main elements: a user terminal (formally denominated as user segment or UES - User Earth Station), such as handheld terminals, portable radios and VSATS, the satellite itself (space segment), which may act either as a signal repeater/hub or as a node\autocite{book1_pp86} that processes and redirects incoming packets and a base station, also known as ground segment or GES - Gateway Earth Station, made of large dishes that amplify the user terminals' signal from an outbound uplink and consequently inbound downlink.

\paragraph{}
Given the fact that satellite networks only support two transmission methods (point to point and broadcast)\autocite{book1_pp28} between UES - satellite, satellite -  satellite (ISL - Inter-Satellite Link) and satellite - GES, their only possible configurations must therefore be SHSS - Single Hop-Single Satellite, DHSS - Double Hop-Single Satellite, MHMS - Multiple Hop-Multiple Satellite and SHMS - Single Hop-Multiple Satellite. For each configuration, a different set of trade-offs must be considered, depending on the intended application.

In remote areas where there is lack of ground communication with the main network, for instance, in a small island or in a rural area, an SHSS configuration would be adopted. In this star topology, all user terminals connect to each other via a single satellite, however, these must use a large antenna in order to transmit to the destination terminal. As a consequence, the power requirements for such devices become much higher than the average UES. This is because the satellite is designed to consume as little power as possible, since all of the energy comes from its solar panels, therefore, all users must deal with the power "deficiency" problem by utilising large antennas. Besides, this topology is designed so that the cost is shifted towards the customer rather than the satellite itself or the GES, which means in the SHSS configuration the space segment is relatively low cost, whereas user segment is of higher cost (compared to DHSS).\\
Despite these issues, the advantage comes from the fact that each packet transmission is done in a single hop, directly from the source terminal $>$ satellite $>$ destination terminal. This means the delay is only one RTT (Round Trip Time). Moreover, implementation should be relatively simple, since there is no interconnection between satellites, no ISL and no handover. The satellite merely acts as a "mirror" that retransmits the signal to its destination. Also, this topology should be immune to ground segment downtimes or security breaches, since it is composed of only user terminals and a single satellite. In the end, it all comes down to the quality of the link (which unfortunately varies during the day) and the size of the user terminals' antennas.\\

On the other hand, when the cost and size of the terminals become too "impractical", a DHSS configuration should be considered instead. An example of this is standard TV broadcast such as DVB-S or in the case where the destination terminal is too far away for any quality link to take place\autocite{book1_pp87} . This means that DHSS covers a much wider area than the previous configuration. Note that the implementation complexity and cost of the satellite in terms of power consumption have not changed as compared to SHSS.
In this topology, all traffic is first redirected to a GES, where it is amplified and retransmitted back through the same satellite and consequently to the destination UES. This means the RTT has doubled, or, in other words, the delay has increased twice as high compared to the SHSS topology. In this case, latency has been sacrificed for a very beneficial trade-off. Now, users can use a very small antenna to connect with other terminals for both RX and TX communications. All of the TX power is now being amplified by the GES. One unfortunate trade-off resultant from the introduction of a single earth segment is the link vulnerability. Since all terminals are connected via a single base station in this topology, if this link goes down, then all of the network is also considered to be down.\\

In the case where the user needs to transmit data over much larger distances, say, from one country to another, a MHMS topology should be used. With this design, all terminals are connected to a complex and transparent satellite-GES network, where an indeterminate amount of hops are taken from source to destination. This interconnection is what is called a transit network. The first disadvantage is the delay. Round trip times are too difficult to predict as a result of the complex nature of the network. Secondly, the earth-segment cost is much higher than the DHSS topology, due to the fact that there is more than one GES being used for the transmission. Moreover, the cost of space-segment is equally higher for the same reason. The advantages, however, come in the form of much wider coverage than the DHSS design, the space segment complexity is also low (since there is no ISL) and the size of the user terminals do not have to deal with any extra TX power. As can be seen, this topology's disadvantages outweigh the advantages (not quantitatively), which leads to the conclusion that this network is best fit for transient communications from one large gateway to another, or in other words, from one country/private organisation to another.\\

Lastly, in critical situations where there are no UES or GES nearby, for instance in the middle of the pacific ocean, an SHMS topology should definitely be adopted. This configuration implements ISL communications between many satellites with global coverage. This means the complexity is much higher due to handover and extra processing required. Also, because there is more than one satellite in the topology, the delay will always be more than one RTT. As a consequence from all this, the cost of space-segment becomes exponentially higher. To make it worse, the UES antennas must be capable of handling transmission in order to compensate for the inevitable power loss. This only leaves the advantages of having global coverage, no earth-segment vulnerabilities and low cost.