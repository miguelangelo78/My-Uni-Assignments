\paragraph{}
Satellite networks are connected with the following three elements: a user terminal, formally named \squotes{user segment} or User Earth Station (UES), such as handheld terminals, portable radios and VSATS; the satellite itself, which makes up the space segment and may act either as a signal repeater or as a node\autocite{book1_pp86} that processes and routes incoming frames; and a base station, also known as \squotes{ground segment} or Gateway Earth Station (GES), made of large satellite dishes that amplify the UES signal from an outbound uplink and consequently inbound downlink.

Considering the fact that satellite networks support only two transmission methods: point to point and broadcast\autocite{book1_pp28} , between UES $\iff$ satellite, satellite $\iff$ satellite (also known as an Inter-Satellite Link [ISL]) and satellite $\iff$ GES, the only possible network configurations are Single Hop-Single Satellite (SHSS), Double Hop-Single Satellite (DHSS), Multiple Hop-Multiple Satellite (MHMS) and Single Hop-Multiple Satellite (SHMS). For each topology, a different set of trade-offs must be considered based on the desired application.

\paragraph{}
In remote areas where there is lack of ground communication with the main network, for instance, in a small island or in a rural area, an SHSS configuration would be adopted. In this mesh topology, all user terminals connect to each other via a single satellite. These must use a large high gain antenna in order to transmit to the destination terminal, hence, the power requirements for such devices are much higher than the average UES.

This condition is explained by the fact that satellites are designed to consume as little power as possible due to its limited power source (i.e. solar panels). Besides, the SHSS topology is purposefully built with the cost shifted towards the customer rather than the satellite itself or the GES (not present in this case). This means the space and earth segments are relative low cost, whereas user segment is of higher cost.
Despite this, the benefit comes in the form of short transmission delays, simplicity and security. Specifically, in this topology a frame is sent and received in a single hop directly from source UES $>$ satellite $>$ destination UES within one Round Trip Time (RTT). In addition, implementation should be relatively simple as there is no interconnection between satellites (ISL). The satellite will merely act as a \squotes{mirror} in order to retransmit the signal to its destination. Also, it should be immune to ground segment downtimes or security breaches since it is composed of user terminals and a single satellite. In the end, it all comes down to the quality of the link (which may unfortunately vary during the day) and the size of the user terminals' antennas.\\

On the other hand, when the cost and size of the UES become too impractical, a DHSS configuration should instead be considered. An example use-case scenario for this topology is standard TV broadcast (DVB-S) or in the situation where the destination terminal is too far to reach and the signal needs to be boosted by a GES\autocite{book1_pp87}. Considering this, DHSS covers a much wider area than the previous topology, i.e. multiple cities or an entire small country.

In this configuration, all traffic is first redirected to a GES, where it is amplified and retransmitted back through the same satellite (or a different one\autocite{book1_pp86}) and consequently to the destination UES. Unfortunately, the RTT has doubled in respect to the SHSS topology; however, this comes with a very beneficial trade-off: users can now use a very small antenna to connect with other terminals for both RX and TX communications since all of the TX power is amplified by the GES. Also, the implementation complexity and cost of the satellite (in terms of design, construction and power consumption) should be the same compared to the SHSS, which could be an advantage to the telecoms company.

One unfortunate consequence of having a single GES is link vulnerability. Since all terminals are connected through a single base station, in the unlucky event of the link going down, the network will completely lose the ability to reach external networks and all of the low gain UES will not be able to communicate with each other. Moreover, if a suspecting user gains physical access to the single GES it is possible for him to compromise the entire DHSS network.

\newpage
In a different setting where the user needs to transmit data over much larger distances, say, from one country to another, a MHMS topology should be used. With this design, all terminals are connected to a complex and transparent satellite-GES network, where an indeterminate amount of hops are taken from source to destination. This interconnection is what is called a transit network.

The first disadvantage is the delay: round trip times are too difficult to predict due to the complex nature of the network. Secondly, the earth-segment cost is much higher than the DHSS topology due to the fact that there is more than one GES being used for the transmission. Also, the cost of the space-segment is equally higher for the same reason. The advantages, however, come in the form of much wider coverage than the DHSS, the space segment complexity is relatively low (since there is no ISL) and the user terminals do not have to compensate for any extra TX power; hence, small antennas are enough to reach the main network. As can be seen, this topology's disadvantages outweigh the advantages (not quantitatively), which leads to the conclusion that this network is best fit for transient communications from one large gateway to another, or from one country/private organisation to another.\\

Lastly, in critical situations where there are no UES or GES nearby, for instance, in the middle of the pacific ocean, an SHMS topology should definitely be adopted. This configuration implements ISL communications and handover between multiple satellites providing global coverage, meaning the space-segment complexity is much higher than the previous configurations. Moreover, because there is more than one satellite in this topology, the delay will always be more than one RTT. Therefore, for all the listed reasons, the cost of the space-segment will be exponentially higher. In addition to this, the users must handle the transmission with large antennas in order to compensate for the inevitable power loss. This only leaves the advantages of having global coverage and no earth-segment vulnerabilities/cost.